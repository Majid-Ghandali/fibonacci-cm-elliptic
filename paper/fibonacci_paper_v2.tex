\overfullrule=5pt

\documentclass[11pt]{amsart}

% ============================================================================
% PACKAGES
% ============================================================================
\usepackage[utf8]{inputenc}
\usepackage[T1]{fontenc}
\usepackage{amsmath}
\usepackage{amssymb}
\usepackage{amsthm}
\usepackage{mathtools}
\usepackage{booktabs}
\usepackage{float}
\usepackage[section]{placeins}
\usepackage{enumerate}

\usepackage[colorlinks=true,linkcolor=black,citecolor=black,urlcolor=black]{hyperref}
\usepackage[nameinlink]{cleveref}
\usepackage[numbers,sort&compress]{natbib}
\usepackage{xurl}
\usepackage[margin=1in]{geometry}

\numberwithin{equation}{section}
\allowdisplaybreaks

% ============================================================================
% THEOREM ENVIRONMENTS
% ============================================================================
\newtheorem{theorem}{Theorem}[section]
\newtheorem{lemma}[theorem]{Lemma}
\newtheorem{proposition}[theorem]{Proposition}
\newtheorem{corollary}[theorem]{Corollary}

\theoremstyle{definition}
\newtheorem{definition}[theorem]{Definition}

\theoremstyle{remark}
\newtheorem{remark}[theorem]{Remark}
\newtheorem{example}[theorem]{Example}

% ============================================================================
% CUSTOM COMMANDS
% ============================================================================
\newcommand{\Fp}{\mathbb{F}_p}
\newcommand{\Fpq}{\mathbb{F}_{p^2}}
\newcommand{\Torus}{\mathcal{T}}
\newcommand{\Legendre}[2]{\left(\frac{#1}{#2}\right)}

% ============================================================================
% METADATA
% ============================================================================
\title{Quadratic Residuosity in Fibonacci Sequences:\\
Arithmetic Structure via CM Elliptic Curves\\
and Twisted Character Sums}

\author{Majid Ghandali}
\address{IRAN, TEHRAN}
\email{majid.ghandali@gmail.com}
\thanks{ORCID: \href{https://orcid.org/0009-0001-1097-1770}{0009-0001-1097-1770}}

\date{\today}

\subjclass[2020]{Primary 11B39, 11G05; Secondary 11T23, 14G10}

\keywords{Fibonacci sequence, quadratic residuosity, elliptic curves, complex
multiplication, Frobenius trace, Sato--Tate distribution, twisted character sums}

% ============================================================================
\begin{document}

\begin{abstract}
Numerical experiments for small inert primes $p$ suggest an apparent bias:
approximately $3/8$ of Fibonacci residues are quadratic non-residues. We prove
that this phenomenon is \emph{transient} rather than asymptotic: the Fibonacci
sequence exhibits square-root cancellation in character sums, yielding asymptotic
equidistribution with density converging to $1/2$.

The key technical contribution is a precise analysis of hybrid character sums on
the norm-one torus $\Torus(\Fp) \subset \Fpq^\times$ via $\ell$-adic cohomology.
We establish non-degeneracy of the relevant Kummer sheaf tensor product and apply
Deligne's Riemann Hypothesis to obtain a uniform $O(\sqrt{p})$ bound independent
of character order. The apparent $3/8$ bias is explained as a geometric artifact
of the trace morphism's image structure, not a property of the Fibonacci orbit
itself.

Our main discovery is an exact arithmetic correspondence with the elliptic curve
$E\colon y^2 = x^3 - 4x$. We prove that the torus character sum equals the
negative Frobenius trace $-a_p(E)$, thereby linking the Fibonacci quadratic bias
to complex multiplication. This reveals that the phenomenon is structurally
governed by the Frobenius trace of a CM elliptic curve, rather than by random
fluctuations.
\end{abstract}

\maketitle
\tableofcontents

% ============================================================================
\section{Introduction}\label{sec:intro}
% ============================================================================

\subsection{Notation and Conventions}

Throughout this paper, we adopt the following notation
(see \cref{tab:notation} for a complete summary).

\begin{table}[ht]
\centering
\begin{tabular}{@{}ll@{}}
\toprule
\textbf{Notation} & \textbf{Description} \\
\midrule
$p$ & Odd prime (inert: $\left(\frac{5}{p}\right) = -1$, i.e.,
       $p \equiv \pm 2 \pmod{5}$). \\
$\mathbb{F}_p$ & Finite field $\mathbb{F}_p = \mathbb{Z}/p\mathbb{Z}$. \\
$\mathbb{F}_{p^2}$ & Quadratic extension of $\mathbb{F}_p$. \\
$\alpha$ & Golden ratio element: $\alpha = \frac{1+\sqrt{5}}{2} \in
            \mathbb{F}_{p^2}$ (root of $x^2-x-1$). \\
$\beta$ & Conjugate root: $\beta = \frac{1-\sqrt{5}}{2} = \sigma(\alpha)$. \\
$F_n$ & Fibonacci number: $F_0=0$, $F_1=1$, $F_{n+2}=F_{n+1}+F_n$. \\
$L_n$ & Lucas number: $L_n = \alpha^n + \beta^n$. \\
$\pi(p)$ & Pisano period: smallest $k>0$ with $(F_k,F_{k+1})\equiv(0,1)\pmod{p}$. \\
$\sigma$ & Frobenius automorphism: $\sigma(\xi) = \xi^p$ on $\mathbb{F}_{p^2}$. \\
$\mathrm{Tr}(\xi)$ & Trace: $\mathrm{Tr}_{\mathbb{F}_{p^2}/\mathbb{F}_p}(\xi)
                      = \xi + \sigma(\xi)$. \\
$\mathrm{N}(\xi)$ & Norm: $\mathrm{N}_{\mathbb{F}_{p^2}/\mathbb{F}_p}(\xi)
                     = \xi \cdot \sigma(\xi)$. \\
$T(\mathbb{F}_p)$ & Norm-one torus: $\{\xi \in \mathbb{F}_{p^2}^\times :
                     \mathrm{N}(\xi)=1\}$. \\
$H_p$ & Fibonacci subgroup: $H_p = \langle\alpha\rangle \subseteq
          T(\mathbb{F}_p)$. \\
$d$ & Index: $d = [T(\mathbb{F}_p) : H_p]$ (algebraic defect). \\
$\psi$ & Trace morphism: $\psi(y) = \mathrm{Tr}(y) = y + y^{-1}$ on
          $T(\mathbb{F}_p)$. \\
$\mathcal{I}_p$ & Fibonacci image: $\mathcal{I}_p = \{F_n \bmod p :
                   n \ge 1\} \subseteq \mathbb{F}_p$. \\
$\chi$ & Legendre symbol (quadratic multiplicative character) on $\mathbb{F}_p^\times$. \\
$\left(\frac{a}{p}\right)$ & Legendre symbol: $1$ (QR), $-1$ (QNR),
                               $0$ ($p \mid a$). \\
$\psi_{\mathrm{add}}$ & Additive character of $\mathbb{F}_p$ (used in Appendix~B only). \\
$S_p$ & Fibonacci character sum: $S_p = \sum_{n=1}^{\pi(p)} \chi(F_n)$. \\
$\delta_p$ & QNR density in $\mathcal{I}_p$. \\
$\omega$ & Multiplicative character on $T(\mathbb{F}_p)$. \\
$\widehat{T}$ & Character group (Pontryagin dual) of $T(\mathbb{F}_p)$. \\
$E$ & CM elliptic curve: $E\colon y^2 = x^3 - 4x$ over $\mathbb{Q}$. \\
$a_p$ & Frobenius trace of $E$: $a_p = p+1-\#E(\mathbb{F}_p)$. \\
$\mathcal{L}_\chi$ & Kummer sheaf associated to character $\chi$. \\
$H^i_c(X,\mathcal{F})$ & $\ell$-adic cohomology with compact support. \\
$\mathrm{Frob}_p$ & Frobenius endomorphism at prime $p$. \\
QR / QNR & Quadratic residue / non-residue modulo $p$. \\
CM & Complex multiplication. \\
\bottomrule
\end{tabular}
\caption{Key notation used throughout the paper.}
\label{tab:notation}
\end{table}

\medskip
\noindent\textbf{Conventions.}
\begin{itemize}
\item We focus exclusively on \emph{inert primes}, i.e., primes $p$ satisfying
  $\left(\frac{5}{p}\right) = -1$.
\item The Pisano period satisfies $\pi(p) = p+1$ for inert primes
  \cite{Wall1960,Renault1996}.
\item All character sums are taken over complete periods unless stated otherwise.
\item By ``equidistribution,'' we mean that the density of QNRs converges to
  $1/2$ as $p \to \infty$.
\end{itemize}

\subsection{Background and Motivation}

The Fibonacci sequence, defined by $F_0=0$, $F_1=1$, $F_{n+2}=F_{n+1}+F_n$,
so that $F_1=1, F_2=1, F_3=2, F_4=3, \ldots$, exhibits rich
arithmetic structure when reduced modulo primes \cite{Wall1960,Koshy2001}.
We index the orbit starting from $n=1$, though the sequence is defined
for all $n \ge 0$.
For a prime $p$, the sequence becomes periodic with period $\pi(p)$ (the
Pisano period), and the set of residues
$\mathcal{I}_p = \{F_n \bmod p : 1 \le n \le \pi(p)\}$
forms a proper subset of $\mathbb{F}_p$.

A natural question concerns the distribution of \emph{quadratic characters}
within $\mathcal{I}_p$: do quadratic residues and non-residues appear with equal
frequency? Numerical experiments reveal a striking phenomenon: for primes
$p \equiv \pm 2 \pmod{5}$ (the \emph{inert} case), approximately $3/8$ of
Fibonacci residues are quadratic non-residues, suggesting a systematic deviation
from uniform distribution. Related distribution questions for Fibonacci numbers
modulo primes have been studied in
\cite{Shparlinski2009,CambreaJavaheri2022,JavaheriKrylov2020}.
While Shparlinski \cite{Shparlinski2009} obtained distribution results
via analytic exponential sum techniques, the present work establishes an
\emph{exact} identity $S_p = -a_p(E)$ linking the quadratic character sum
to the Frobenius trace of a CM elliptic curve, providing a sharper and
more structural explanation.

Our main result shows that the observed bias is a transient artifact of the
torus folding, and admits a complete geometric explanation via Frobenius traces
(see \cref{thm:asymptotic-equidistribution,thm:main-formal}).

\begin{theorem}[Asymptotic Equidistribution]\label{thm:asymptotic-equidistribution}
Let $p > 5$ be a prime with $\left(\frac{5}{p}\right) = -1$, and let $\chi$
denote the quadratic character modulo $p$ \cite{IrelandRosen1990}. Define the
character sum over the Fibonacci orbit
\[
  S_p := \sum_{n=1}^{\pi(p)} \chi(F_n \bmod p).
\]
Then $S_p = O(\sqrt{p})$, uniformly in $p$. Consequently, the empirical
density of quadratic non-residues satisfies
\[
  \delta_p := \frac{1}{\pi(p)}
  \#\{n \le \pi(p) : \chi(F_n) = -1\}
  = \frac{1}{2} + O(p^{-1/2}),
\]
establishing asymptotic equidistribution.
\end{theorem}

\subsection{The Apparent \texorpdfstring{$3/8$}{3/8} Phenomenon}

The observed $3/8$ density arises from a \emph{geometric} constraint, not from
any inherent bias in the Fibonacci dynamics. When $p$ is inert, the Fibonacci
sequence embeds into the norm-one torus
\[
  T(\mathbb{F}_p) = \{y \in \mathbb{F}_{p^2}^\times : y^{p+1} = 1\}
\]
via Binet's formula \cite{Koshy2001}. The trace morphism $\psi(y) = y + y^{-1}$
induces a $2$-to-$1$ folding, causing the image to occupy asymptotically a
subset of $\mathbb{F}_p$ of relative size $3/4$. If quadratic characters were
uniformly distributed within this image, one would expect
\[
  \mathcal{D} = \frac{3}{4} \cdot \frac{1}{2} = \frac{3}{8}
\]
of $\mathbb{F}_p$ to consist of Fibonacci quadratic non-residues.

In \cref{sec:geometric} we make this heuristic precise by realizing the Fibonacci
orbit as a dynamical system on a union of conics and a genus-$1$ curve. This
geometric picture explains both the emergence of the $3/8$ bias for finite $p$
and its eventual disappearance in the asymptotic regime.

\subsection{Outline of Proof Strategy}

The proof combines four complementary ingredients:
\begin{enumerate}[(i)]
\item \textbf{Algebraic reformulation.} Using Binet's formula \cite{Koshy2001},
  we express Fibonacci residues as traces of powers of
  $\alpha = (1+\sqrt{5})/2$ in $\mathbb{F}_{p^2}$. The character sum reduces to
  a sum over a cyclic subgroup $H_p \le T(\mathbb{F}_p)$
  (see \cref{prop:binet-torus}).
\item \textbf{Geometric realization.} The Fibonacci orbit is realized as a
  dynamical system on a union of conics and a genus-$1$ curve, giving a
  precise geometric derivation of the apparent $3/8$ density law
  (\cref{sec:geometric}).
\item \textbf{Character decomposition.} We employ Fourier analysis on the torus
  \cite{LidlNiederreiter1997}: the indicator function of $H_p$ decomposes as a
  sum over multiplicative characters annihilating $H_p$. This transforms $S_p$
  into a hybrid sum involving products of two characters.
\item \textbf{Weil bound via cohomology.} The hybrid sums are interpreted as
  traces of Frobenius on rank-$1$ $\ell$-adic sheaves \cite{Katz1988,Serre1968}.
  Using Deligne's theorem \cite{Deligne1974} and a careful non-degeneracy
  analysis, we establish uniform $O(\sqrt{p})$ bounds independent of
  character order.
\end{enumerate}

\subsection{Organization}

\cref{sec:geometric} develops the geometric realization of the Fibonacci orbit on
norm level sets of the quadratic extension and derives the geometric $3/8$ density
law. \cref{sec:characters} analyzes character sums via Fourier decomposition and
proves the non-degeneracy of hybrid characters. \cref{sec:weil} establishes the
Weil bound and discovers the elliptic correspondence.
\cref{sec:main-proof} completes the proof of \cref{thm:main-formal}.
\cref{sec:numerics} presents numerical evidence and discusses the CM Sato--Tate
distribution.

\subsection{Statement of Main Result}

\begin{theorem}[Main Result]\label{thm:main-formal}
Let $p > 5$ be a prime with $\left(\frac{5}{p}\right) = -1$. Let
$\chi\colon \mathbb{F}_p^\times \to \{\pm 1\}$ denote the quadratic character,
and let $\pi(p) = p+1$ denote the Pisano period \cite{Wall1960}. Define the
character sum
\[
  S_p := \sum_{n=1}^{\pi(p)} \chi(F_n \bmod p).
\]
Then there exists an absolute constant $C > 0$ such that
$|S_p| \le C\sqrt{p}$ for all inert primes $p > 5$.
Moreover:
\begin{enumerate}[(i)]
\item $C$ can be taken explicitly as $C = 2$.
\item The empirical density of quadratic non-residues satisfies
\[
  \delta_p := \frac{1}{\pi(p)}
  \#\{1 \le n \le \pi(p) : \chi(F_n) = -1\}
  = \frac{1}{2} + O(p^{-1/2}).
\]
\item \textbf{Elliptic correspondence:} For the CM elliptic curve
  $E\colon y^2 = x^3 - 4x$ \cite{Silverman2009,Silverman1994},
\[
  \sum_{t \in \mathbb{F}_p} \chi(t^3 - 4t) = -a_p(E),
\]
where $a_p$ is the Frobenius trace.
\end{enumerate}
\end{theorem}

\begin{remark}[Support vs.\ Dynamics]
The geometric $3/8$ law (\cref{sec:geometric,eq:density-3-8}) describes the size of the trace
support. The character sum analysis (\cref{sec:characters,sec:weil}) establishes
equidistribution of the orbit within this support, thereby proving
\cref{thm:main-formal}.
\end{remark}

% ============================================================================
\section{Geometric Realization of the Fibonacci Orbit}
\label{sec:geometric}
% ============================================================================

\subsection{Algebraic Setup over Inert Primes}

Let $p > 5$ be an inert prime in $\mathbb{Q}(\sqrt{5})$, i.e.,
$\left(\frac{5}{p}\right) = -1$.
Then $\sqrt{5} \notin \mathbb{F}_p$ and the field extension
$\mathbb{F}_p(\sqrt{5}) \cong \mathbb{F}_{p^2}$ \cite{LidlNiederreiter1997}.
Define the golden ratio element
\[
\alpha = \frac{1+\sqrt{5}}{2} \in \mathbb{F}_{p^2}.
\]
Since $\alpha^2 = \alpha + 1$, the classical Binet formulas hold in
$\mathbb{F}_{p^2}$ \cite{Koshy2001}:
\[
F_n = \frac{\alpha^n - \alpha^{-n}}{\sqrt{5}},
\qquad
L_n = \alpha^n + \alpha^{-n}.
\]

\subsection{Torus Structure and Frobenius Action}

The norm-one algebraic torus over $\mathbb{F}_p$ is
\[
T(\mathbb{F}_p) = \{ \xi \in \mathbb{F}_{p^2}^\times :
\mathrm{N}_{\mathbb{F}_{p^2}/\mathbb{F}_p}(\xi) = 1 \}.
\]
This group is cyclic of order $p+1$ \cite{LidlNiederreiter1997}. More generally,
for $\varepsilon \in \{\pm 1\}$, define the norm level sets:
\[
T_\varepsilon = \{ \xi \in \mathbb{F}_{p^2}^\times :
\mathrm{N}_{\mathbb{F}_{p^2}/\mathbb{F}_p}(\xi) = \varepsilon \}.
\]
Here $T_1$ is the standard norm-one torus, while $T_{-1}$ is a principal
homogeneous space (torsor) over $T_1$.

\begin{lemma}[Frobenius Action and Norm Structure]
\label{lem:frobenius-action}
Let $p > 5$ be a prime inert in $\mathbb{Q}(\sqrt{5})$, i.e., $p \equiv \pm 2 \pmod{5}$.
Then for $\alpha = \frac{1+\sqrt{5}}{2} \in \mathbb{F}_{p^2}$ we have
\[
\mathrm{N}(\alpha) = \alpha^{p+1} = -1,
\qquad
\alpha^p = -\alpha^{-1}.
\]
More generally, for any $\xi \in T_\varepsilon$,
\[
\xi^p = \varepsilon\,\xi^{-1}.
\]
In particular, $\alpha \in T_{-1}$.
\end{lemma}

\begin{proof}
Since $p$ is inert in $\mathbb{Q}(\sqrt{5})$, the prime $p$ does not split in
$\mathbb{Z}[\frac{1+\sqrt{5}}{2}]$, and $\mathbb{F}_{p^2} = \mathbb{F}_p(\sqrt{5})$.
The Frobenius automorphism $\sigma \in \mathrm{Gal}(\mathbb{F}_{p^2}/\mathbb{F}_p)$
is the unique non-trivial element of the Galois group and acts as
$\sigma(\xi) = \xi^p$.
Since $\sqrt{5} \notin \mathbb{F}_p$, Frobenius must exchange the two roots
$\pm\sqrt{5}$ of $x^2 - 5$, giving $(\sqrt{5})^p = -\sqrt{5}$.
Therefore
\[
\alpha^p
= \left(\frac{1+\sqrt{5}}{2}\right)^p
= \frac{1 + (\sqrt{5})^p}{2}
= \frac{1 - \sqrt{5}}{2} = \beta.
\]
A direct computation confirms
\[
\alpha \cdot \alpha^p
= \frac{(1+\sqrt{5})(1-\sqrt{5})}{4}
= \frac{1-5}{4}
= -1,
\]
so $\mathrm{N}(\alpha) = \alpha^{p+1} = -1$,
and multiplying both sides by $\alpha^{-1}$ yields $\alpha^p = -\alpha^{-1}$.

For general $\xi \in T_\varepsilon$, by definition $\xi^{p+1} = \varepsilon$.
Dividing both sides by $\xi$ gives $\xi^p = \varepsilon\,\xi^{-1}$.
\end{proof}

\begin{remark}[Galois Action on Characters]
\label{rem:galois-characters}
The Frobenius identity $\xi^p = \varepsilon\,\xi^{-1}$ describes the action
of the non-trivial element $\sigma \in \mathrm{Gal}(\mathbb{F}_{p^2}/\mathbb{F}_p)$
on the norm torus $T_\varepsilon$.

For $\varepsilon = 1$, this action restricts to $T_1 = \Torus(\Fp)$ and
induces an action on the character group: if
$\omega\colon \Torus(\Fp) \to \mathbb{C}^\times$ is a character, then
\[
\sigma^*\omega\colon \xi \mapsto \omega(\xi^p)
= \omega(\xi^{-1}) = \omega(\xi)^{-1} = \omega^{-1}(\xi).
\]
That is, Frobenius acts on $\widehat{\Torus(\Fp)}$ by inversion:
\[
\sigma^* \colon \omega \longmapsto \omega^{-1}.
\]
In particular, the only Frobenius-fixed characters are $\omega = \mathbf{1}$
and (when $2 \mid p+1$) the unique character of order $2$.

This inversion symmetry implies that the non-trivial character sums
appearing in \cref{sec:characters} arise in conjugate pairs
$T(\omega)$ and $T(\omega^{-1}) = \overline{T(\omega)}$,
a fact that underlies the square-root cancellation established in
\cref{sec:weil}.

The Fibonacci sequence modulo $p$ is not an arbitrary recurrence; it is the
projection of a cyclic subgroup oscillating between $T_1$ and its torsor $T_{-1}$.
This geometric structure enables the application of $\ell$-adic methods
\cite{Serre1968,Katz1988}.
\end{remark}

\subsection{Torus--Conic Correspondence}

From the Binet formulas we obtain the algebraic identity
$L_n^2 - 5F_n^2 = 4(-1)^n$.
For a fixed parity of $n$, this defines an affine conic over $\mathbb{F}_p$.

\begin{proposition}[Torus--Conic Isomorphism]
\label{prop:isomorphism}
Let $p$ be inert in $\mathbb{Q}(\sqrt{5})$. Define the map
\[
\Psi\colon \mathbb{F}_{p^2}^\times \longrightarrow \mathbb{A}^2(\mathbb{F}_p),
\qquad
\Psi(\xi) = \left( \mathrm{Tr}(\xi),\; \frac{\xi - \xi^p}{\sqrt{5}} \right).
\]
Then for each $\varepsilon \in \{\pm 1\}$, the restriction of $\Psi$ induces a
bijection between $T_\varepsilon$ and the affine conic
\[
\mathcal{C}_\varepsilon\colon \quad x^2 - 5y^2 = 4\varepsilon.
\]
In particular, $\Psi(\alpha^n) = (L_n, F_n)$, and the identity
$L_n^2 - 5F_n^2 = 4(-1)^n$ is the image under $\Psi$ of the norm relation
$\mathrm{N}(\alpha^n) = (-1)^n$.
\end{proposition}

\begin{proof}
Write $\xi = \frac{x + y\sqrt{5}}{2}$ with $x, y \in \mathbb{F}_p$. Then
$\xi^p = \frac{x - y\sqrt{5}}{2}$, yielding
$\mathrm{Tr}(\xi) = x$ and $(\xi - \xi^p)/\sqrt{5} = y$.
The norm is
\[
\mathrm{N}(\xi) = \xi \cdot \xi^p = \frac{x^2 - 5y^2}{4},
\]
so $\mathrm{N}(\xi) = \varepsilon$ is equivalent to $x^2 - 5y^2 = 4\varepsilon$.
The map $\Psi$ is invertible via $\xi = \frac{x + y\sqrt{5}}{2}$.
For $\xi = \alpha^n$, the Binet formulas give $\Psi(\alpha^n) = (L_n, F_n)$,
and \cref{lem:frobenius-action} with multiplicativity of the norm gives
$\mathrm{N}(\alpha^n) = (-1)^n$.
\end{proof}

\begin{corollary}[Trace Cardinality]
\label{cor:trace-cardinality}
The trace projection $\mathrm{pr}\colon T_\varepsilon \to \mathbb{F}_p$,
$\mathrm{pr}(\xi) = \mathrm{Tr}(\xi)$, satisfies
\[
|\mathrm{Tr}(T_1)| = \frac{p+3}{2},
\qquad
|\mathrm{Tr}(T_{-1})| = \frac{p + 2 + \left(\frac{-1}{p}\right)}{2},
\]
where $\left(\frac{-1}{p}\right)$ is the Legendre symbol
\cite{IrelandRosen1990}.
\end{corollary}

\begin{proof}
For $T_1$: since $\xi^p = \xi^{-1}$, the trace $\mathrm{Tr}(\xi) = \xi + \xi^{-1}$
is invariant under $\xi \leftrightarrow \xi^{-1}$. The $p+1$ elements of $T_1$
pair up under this involution except for the two fixed points $\xi = \pm 1$
(where $\xi = \xi^{-1}$), giving
\[
|\mathrm{Tr}(T_1)| = \frac{(p+1) - 2}{2} + 2 = \frac{p+3}{2}.
\]

For $T_{-1}$: since $\xi^p = -\xi^{-1}$, we have
$\mathrm{Tr}(\xi) = \xi + \xi^p = \xi - \xi^{-1}$.
The map $\xi \mapsto -\xi^{-1}$ is an involution on $T_{-1}$
(since $\mathrm{N}(-\xi^{-1}) = \mathrm{N}(\xi)^{-1} \cdot (-1)^2 = -1$).
Fixed points satisfy $\xi = -\xi^{-1}$, i.e., $\xi^2 = -1$.
Such solutions exist in $T_{-1}$ if and only if $-1$ is a quadratic residue
mod $p$, i.e., $p \equiv 1 \pmod{4}$, contributing $\bigl(\tfrac{-1}{p}\bigr)+1$
fixed points ($0$ or $2$). Therefore the $p+1$ elements of $T_{-1}$ give
\[
|\mathrm{Tr}(T_{-1})|
= \frac{(p+1) - \bigl(1+\bigl(\tfrac{-1}{p}\bigr)\bigr)}{2}
  + \bigl(1+\bigl(\tfrac{-1}{p}\bigr)\bigr)
= \frac{p + 2 + \bigl(\tfrac{-1}{p}\bigr)}{2}. \qedhere
\]
\end{proof}

\begin{remark}[The Arithmetic Bridge]
\label{rem:arithmetic-bridge}
\cref{prop:isomorphism} establishes that statistical properties of $\{L_n\}$ may
be interpreted as distributional properties of $x$-coordinates of rational points
on $\mathcal{C}_+ \cup \mathcal{C}_-$. The observed density
$\mathcal{D} \approx 3/8$ emerges as the limiting proportion of points
$(x,y) \in \mathcal{C}_+ \cup \mathcal{C}_-$ for which $x$ is a quadratic
non-residue.
\end{remark}

\subsection{Quadratic Lift and the Genus-1 Transition}
\label{subsec:quadratic-lift}

The condition that the trace $x = L_n$ is a perfect square, $x = z^2$, induces
a quadratic lift of the conic $\mathcal{C}_\varepsilon$:
\begin{equation}\label{eq:hyperelliptic-eqn}
z^4 - 5y^2 = 4\varepsilon
\quad \Longrightarrow \quad
y^2 = \frac{z^4 - 4\varepsilon}{5}.
\end{equation}
This defines a hyperelliptic curve $\mathcal{H}_\varepsilon$ of degree $4$.

\begin{lemma}[Smoothness and Genus of the Lift]
\label{lem:smoothness}
For every inert prime $p > 5$, the projective closure of
\[
\mathcal{H}_\varepsilon^{\mathrm{aff}}\colon\;
y^2 = \frac{z^4 - 4\varepsilon}{5}
\]
is smooth of genus
\[
g = \left\lfloor\frac{4-1}{2}\right\rfloor = 1.
\]
\end{lemma}

\begin{proof}
Since $p > 5$ we have $5 \in \mathbb{F}_p^\times$. Set
$F(z,y) = y^2 - (z^4 - 4\varepsilon)/5$. Affine singular points require
$\partial_y F = 2y = 0$ and $\partial_z F = -4z^3/5 = 0$, forcing $y = z = 0$.
But $(0,0)$ does not satisfy $F = 0$ since $-4\varepsilon/5 \ne 0$.
Thus no affine singularity exists. The polynomial $z^4 - 4\varepsilon$ is
separable over $\overline{\mathbb{F}}_p$ (its derivative $4z^3$ shares no root
with it), so the projective hyperelliptic model has two non-singular points at
infinity. The genus formula for a separable degree-$4$ polynomial gives $g = 1$.
\end{proof}

\subsection{Trace Support and Character Sum Estimates}
\label{subsec:character-sums}

Let $I_\varepsilon = \mathrm{pr}(T_\varepsilon)$ be the set of traces from $T_\varepsilon$,
where $\mathrm{pr}\colon T_\varepsilon \to \mathbb{F}_p$ denotes the trace projection
$\mathrm{pr}(\xi) = \mathrm{Tr}(\xi)$.
An element $x \in \mathbb{F}_p$ belongs to $I_\varepsilon$ if and only if
$x^2 - 4\varepsilon$ is a quadratic residue (or zero) in $\mathbb{F}_p$.

\begin{lemma}[Asymptotic Intersection of Trace Images]
\label{lem:intersection-size}
For inert primes $p > 5$,
\[
|I_1 \cap I_{-1}| = \frac{p}{4} + O(\sqrt{p}).
\]
\end{lemma}

\begin{proof}
Using the quadratic character $\chi$ \cite{IrelandRosen1990},
\[
\mathbf{1}_{I_\varepsilon}(x) = \frac{1+\chi(x^2-4\varepsilon)}{2} + O(1/p),
\]
with error $O(1)$ from at most two exceptional values. Therefore
\begin{align*}
|I_1 \cap I_{-1}|
&= \frac{1}{4}\sum_{x \in \mathbb{F}_p}
   \Bigl(1 + \chi(x^2-4) + \chi(x^2+4)
   + \chi\bigl((x^2-4)(x^2+4)\bigr)\Bigr) + O(1)\\
&= \frac{p}{4} + \frac{1}{4}\sum_{x \in \mathbb{F}_p}\chi(x^4-16) + O(1).
\end{align*}
Since $x^4-16$ is a separable polynomial of degree $4$ for $p > 5$, the Weil
bound \cite{IrelandRosen1990,Bombieri1966} gives
$\bigl|\sum_{x}\chi(x^4-16)\bigr| \le 3\sqrt{p}$, yielding the claim.
\end{proof}

\begin{proposition}[Total Trace Support]
\label{prop:total-support}
$|I_{\mathrm{tot}}| = |I_1 \cup I_{-1}| = \tfrac{3}{4}p + O(\sqrt{p})$.
\end{proposition}

\begin{proof}
By \cref{cor:trace-cardinality},
$|I_1| = \frac{p+3}{2}$ and
$|I_{-1}| = \frac{p + 2 + \bigl(\tfrac{-1}{p}\bigr)}{2}$,
so $|I_\varepsilon| = \frac{p}{2} + O(1)$ for both $\varepsilon$.
By inclusion--exclusion and \cref{lem:intersection-size},
\[
|I_{\mathrm{tot}}|
= |I_1| + |I_{-1}| - |I_1 \cap I_{-1}|
= \frac{p+3}{2} + \frac{p}{2} + O(1) - \frac{p}{4} - O(\sqrt{p})
= \frac{3}{4}p + O(\sqrt{p}). \qedhere
\]
\end{proof}

\subsection{Refinement via Hasse--Weil}
\label{subsec:three-eighths}

For a genus-$1$ curve, the Hasse--Weil bound asserts
\cite{Silverman2009}:
\[
\bigl|\#\mathcal{H}_\varepsilon(\mathbb{F}_p) - (p+1)\bigr| \le 2\sqrt{p},
\]
where $\mathcal{H}_\varepsilon$ is the quadratic lift defined in
\cref{eq:hyperelliptic-eqn}.

\begin{remark}[Orbit Largeness and GRH]
\label{rem:artin-grh}
The defect index $d = [T(\mathbb{F}_p) : H_p]$ governs the number of
characters appearing in the Fourier decomposition of Step~2 of the proof.
If $d$ were as large as $p+1$ (i.e., $H_p$ were trivial), each of the
$p+1$ individual terms $T(\omega)$ would be bounded by $2\sqrt{p}$, but
the sum would only give $|S_p| \le (p+1) \cdot 2\sqrt{p}/( p+1) = 2\sqrt{p}$
regardless — so the bound $C = 2$ is unconditional.

However, a non-trivial lower bound on
$\operatorname{ord}(\alpha) = |H_p| = (p+1)/d$, specifically
$\operatorname{ord}(\alpha) \gg p^{1/2}$ (i.e., $d \ll p^{1/2}$),
would sharpen the equidistribution rate in Part~(ii) to
$\delta_p = \tfrac{1}{2} + O(d/p)$ rather than $O(p^{-1/2})$.
It is unknown whether this holds for a density-one set of inert primes.
Conditional on the Generalised Riemann Hypothesis, density-one statements
for multiplicative order problems on algebraic tori are expected from
the theory of Galois representations \cite{Serre1968}.
\end{remark}

Combining the torus folding (trace image of relative size $3/4$,
see \cref{prop:total-support}) with the genus-$1$ quadratic lift
(\cref{lem:smoothness}; factor of $1/2$ from quadratic residue symmetry):
\begin{equation}\label{eq:density-3-8}
\mathcal{D}_{\mathrm{geom}} = \frac{3}{4} \cdot \frac{1}{2} = \frac{3}{8}.
\end{equation}
This $3/8$ law emerges as an intrinsic consequence of the arithmetic geometry of
the Fibonacci torsors, with fluctuations bounded by the spectral gap of the
associated Frobenius endomorphism.

% ============================================================================
\section{Character Sums and Fourier Decomposition}
\label{sec:characters}
% ============================================================================

\subsection{Dual Group Structure}

\begin{proposition}[Dual Group of the Torus]
\label{prop:dual-torus}
Let $\Torus(\Fp)$ denote the norm-one torus. Then:
\begin{enumerate}[(i)]
\item $\Torus(\Fp)$ is cyclic of order $p+1$ \cite{LidlNiederreiter1997}.
\item The character group $\widehat{\Torus}$ is isomorphic to
  $\mathbb{Z}/(p+1)\mathbb{Z}$ by Pontryagin duality.
\item For any subgroup $H \le \Torus(\Fp)$ of index $d = [\Torus : H]$,
\[
\mathbf{1}_H(y) = \frac{1}{d} \sum_{\substack{\omega \in \widehat{\Torus} \\
  \omega^d = 1}} \omega(y).
\]
\end{enumerate}
\end{proposition}

\begin{proof}
Part~(i) follows from the structure theory of cyclic groups over finite fields
\cite{LidlNiederreiter1997}. Part~(ii) is Pontryagin duality for finite cyclic
groups. For~(iii), the orthogonality relations for characters give
\[
\mathbf{1}_H(y) = \frac{1}{|\Torus|} \sum_{\omega \in \widehat{\Torus}}
\left( \sum_{h \in H} \overline{\omega}(h) \right) \omega(y).
\]
The inner sum vanishes unless $\omega|_H \equiv 1$; since $\Torus$ is cyclic,
the annihilator of $H$ consists of characters of order dividing $d$.
\end{proof}

\subsection{Hybrid Character Sums}

Define the trace morphism $\psi\colon \Torus \to \Fp$ by
$\psi(y) = y + y^{-1}$.

\begin{proposition}[Binet--Torus Identification]
\label{prop:binet-torus}
Let $\alpha \in \mathbb{F}_{p^2}$ be a root of $x^2 - x - 1$,
and set $H_p = \langle \alpha \rangle \subseteq \Torus(\Fp)$.
Then for all $1 \le n \le \pi(p)$,
\[
F_n = \frac{\alpha^n - \alpha^{-n}}{\alpha - \alpha^{-1}},
\]
and consequently
\[
\chi(F_n) = \chi(\alpha - \alpha^{-1})^{-1} \cdot \chi(\alpha^n - \alpha^{-n}).
\]
Since $\chi$ is completely multiplicative and
$c_p := \chi(\alpha - \alpha^{-1}) \in \{\pm 1\}$ is a constant depending only on $p$,
\[
S_p = c_p \sum_{y \in H_p} \chi(y - y^{-1}).
\]
Moreover, using the identity $(y - y^{-1})^2 = (y + y^{-1})^2 - 4 = \psi(y)^2 - 4$,
\[
\chi(y - y^{-1})^2 = \chi\!\bigl(\psi(y)^2 - 4\bigr),
\]
so each term is determined by the trace morphism $\psi(y) = y + y^{-1}$.
In particular, the sum $S_p$ is determined up to the sign $c_p$ by the
values $\{\psi(y) : y \in H_p\}$.
\end{proposition}

\begin{proof}
The Binet formula $F_n = (\alpha^n - \beta^n)/(\alpha - \beta)$ holds in
$\mathbb{F}_{p^2}$ since $\alpha$ satisfies $\alpha^2 = \alpha + 1$
\cite{Koshy2001}. Since $\alpha\beta = -1$ (the product of roots of $x^2-x-1$),
we have $\beta = -\alpha^{-1}$, so $\alpha - \beta = \alpha + \alpha^{-1}$
and $\alpha^n - \beta^n = \alpha^n - (-\alpha^{-1})^n$
(see \cref{lem:frobenius-action} for the norm relation $\alpha^{p+1}=-1$).
Since $\chi$ is completely multiplicative,
$\chi(F_n) = \chi(\alpha - \alpha^{-1})^{-1} \cdot \chi(\alpha^n - \alpha^{-n})$.
Summing over $n = 1, \ldots, p+1$ (one full period) gives
$S_p = c_p \sum_{y \in H_p} \chi(y - y^{-1})$.
The identity $(y - y^{-1})^2 = \psi(y)^2 - 4$ follows from direct expansion.
\end{proof}

\begin{remark}
The sign $c_p = \chi(\alpha - \alpha^{-1}) = \chi(\sqrt{5})$ depends on whether
$5$ is a QR mod $p$. Since $p$ is inert, $\bigl(\tfrac{5}{p}\bigr) = -1$,
so $c_p = -1$ for all primes in our family. This sign is absorbed into the
$O(\sqrt{p})$ bound and does not affect equidistribution.
\end{remark}

\begin{lemma}[Character Sum Decomposition]
\label{lem:decomposition}
With $d = [\Torus : H_p]$,
\[
S_p = \sum_{n=1}^{\pi(p)} \chi(F_n)
= \frac{1}{d} \sum_{\substack{\omega \in \widehat{\Torus} \\ \omega^d = 1}}
\sum_{y \in \Torus(\Fp)} \omega(y)\, \chi(\psi(y)).
\]
\end{lemma}

\begin{proof}
By \cref{prop:binet-torus}, $S_p = c_p \sum_{y \in H_p} \chi(y - y^{-1})$.
Since $(y - y^{-1})^2 = \psi(y)^2 - 4$, the quadratic character satisfies
$\chi(\psi(y)^2 - 4) = 1$ for all $y \neq \pm 1$ (as $\psi(y)^2 - 4$
is a perfect square in $\mathbb{F}_p$). Therefore the Legendre evaluation
depends only on the image under the $2$-to-$1$ map
$\psi\colon \Torus(\Fp) \to I_\varepsilon$,
and the sum over $H_p$ folds into a sum over its trace image.
Applying \cref{prop:dual-torus}(iii) to decompose $\mathbf{1}_{H_p}$
and summing $\chi(\psi(y))$ over $\Torus(\Fp)$ yields the stated formula.
\end{proof}

\subsection{Non-degeneracy Analysis}

\begin{proposition}[Non-degeneracy of Hybrid Character]
\label{prop:non-degenerate}
Let $\omega$ be a non-trivial multiplicative character on $\Torus(\Fp)$. Then
$f(y) = \omega(y)\,\chi(\psi(y))$ is non-constant on $\Torus(\Fp)$.
\end{proposition}

\begin{proof}
Suppose $f$ is constant. Since $\chi$ takes values in $\{\pm 1\}$, constancy
forces $\chi(\psi(y)) = \omega(y)^{-1}$ for all $y$, hence $\omega$ has order
dividing $2$. The unique non-trivial order-$2$ character on $\Torus(\Fp)$ is
$\omega_2(y) = y^{(p+1)/2}$. The identity $\chi(y+y^{-1}) = \omega_2(y)$
would yield
\[
\left(y + y^{-1}\right)^{(p-1)/2} = y^{(p+1)/2}
\]
for all $y \in \Torus(\Fp)$. The difference $\Delta(y)$ of left and right sides
is a non-zero Laurent polynomial of degree $p-1 < p+1 = |\Torus(\Fp)|$ that
vanishes on every element of the cyclic group $\Torus(\Fp)$---a contradiction.
\end{proof}

% ============================================================================
\section{Weil Bound via \texorpdfstring{$\ell$}{l}-Adic Cohomology}
\label{sec:weil}
% ============================================================================

\subsection{Square-Freeness of the Trace Morphism}

\begin{lemma}[Square-Freeness]
\label{lem:square-free}
The function $\psi(y) = y + y^{-1} = (y^2+1)/y$ is not a perfect square in the
function field of the torus.
\end{lemma}

\begin{proof}
If $\psi = h^2$ for some rational function $h$, then every zero and pole of
$\psi$ must occur with even multiplicity. But $\psi$ has simple poles at $y=0$
and $y=\infty$---contradiction.
\end{proof}

\subsection{Hybrid Character Sum Bound}

\begin{lemma}[Hybrid Bound]
\label{lem:hybrid-bound}
Let $\omega$ be a character on $\Torus(\Fp)$ of order $d \mid (p+1)$. Then
\[
\left| \sum_{y \in \Torus(\Fp)} \omega(y)\,\chi(\psi(y)) \right| \le 2\sqrt{p}.
\]
\end{lemma}

\begin{proof}
Consider the $\ell$-adic sheaf $\mathcal{F} := \mathcal{L}_\omega \otimes
\psi^*\mathcal{L}_\chi$ on $\mathbb{G}_m$ \cite{Katz1988}.

\medskip
\noindent\textbf{Non-degeneracy.}
By \cref{lem:square-free}, $\psi(y) = y + y^{-1}$ is not a perfect square
in the function field of $\mathbb{G}_m$. This ensures that the pullback
$\psi^*\mathcal{L}_\chi$ is a geometrically non-trivial rank-$1$ local system,
so $\mathcal{F}$ is not geometrically isomorphic to $\mathcal{L}_\omega$ alone.

\medskip
\noindent\textbf{Ramification.}
The equation $x + x^{-1} = t$ has discriminant $\Delta = t^2 - 4$, so
$\mathcal{F}$ is tamely ramified exactly at $t = \pm 2$ (i.e., $x = \pm 1$),
with Swan conductor $\mathrm{sw}(\mathcal{F}) = 0$ \cite{Katz1988}.

\medskip
\noindent\textbf{Cohomology.}
We apply the Grothendieck--Ogg--Shafarevich formula in the form
\cite{Katz1988}:
\[
\chi_c(\mathbb{G}_m,\mathcal{F})
= \mathrm{rank}(\mathcal{F})\cdot\chi_c(\mathbb{G}_m,\mathbb{Q}_\ell)
  - \sum_{x \in \{\pm 1\}} \mathrm{cond}_x(\mathcal{F}),
\]
where $\mathrm{cond}_x(\mathcal{F}) = \mathrm{sw}_x(\mathcal{F})
+ \dim(\mathcal{F}_{\bar x}^{I_x=1} \text{ drop})$
is the local conductor at each ramification point.
We have:
\begin{itemize}
\item $\mathrm{rank}(\mathcal{F}) = 1$;
\item $\chi_c(\mathbb{G}_m, \mathbb{Q}_\ell) = 0$
  (since $\mathbb{G}_m = \mathbb{P}^1 \setminus \{0,\infty\}$
  has Euler characteristic $2 - 2 = 0$);
\item ramification is tame at $y = \pm 1$ with
  $\mathrm{sw}_{\pm 1}(\mathcal{F}) = 0$
  and inertia-invariant dimension drop $= 1$ at each point;
  hence $\mathrm{cond}_{\pm 1}(\mathcal{F}) = 1$.
\end{itemize}
Therefore:
\[
\chi_c(\mathbb{G}_m,\mathcal{F}) = 1 \cdot 0 - (1 + 1) = -2.
\]
Since $\omega$ is non-trivial and $\mathcal{L}_\omega$ is a non-constant
rank-$1$ local system, $H^0_c(\mathbb{G}_m,\mathcal{F}) = 0$.
By Poincar\'{e} duality for $\mathbb{G}_m \subsetneq \mathbb{P}^1$,
$H^2_c(\mathbb{G}_m, \mathcal{F}) = 0$.
From the Euler characteristic relation
$\chi_c = \dim H^0_c - \dim H^1_c + \dim H^2_c$:
\[
-2 = 0 - \dim H^1_c(\mathbb{G}_m,\mathcal{F}) + 0,
\qquad\Longrightarrow\qquad
\dim H^1_c(\mathbb{G}_m,\mathcal{F}) = 2.
\]
Each $\mathcal{L}_\omega$ and $\mathcal{L}_\chi$ is a Kummer sheaf
pointwise pure of weight $0$; by \cite{Deligne1974}, $H^1_c$ is then
pure of weight $0 + 1 = 1$, so every Frobenius eigenvalue has
absolute value $p^{1/2}$.

\medskip
\noindent\textbf{Deligne bound.}
Deligne's Riemann Hypothesis \cite{Deligne1974} gives Frobenius eigenvalues of
absolute value $\sqrt{p}$, hence
\[
\left|\sum_{y \in \Torus(\Fp)} \omega(y)\,\chi(\psi(y))\right|
= \left|\mathrm{Tr}(\mathrm{Frob}_p \mid H^1_c(\mathbb{G}_m,\mathcal{F}))\right|
\le 2\sqrt{p}. \qedhere
\]
\end{proof}

\subsection{Elliptic Reduction}

\begin{lemma}[Geometric Origin of the Cubic]
\label{lem:cubic-origin}
Let $\psi\colon \Torus(\Fp) \to \Fp$ be the trace morphism $\psi(y) = y + y^{-1}$,
and set $t = \psi(y)$.
Define the anti-trace $z(y) = y - y^{-1}$. Then:
\begin{enumerate}[(i)]
\item $z(y)^2 = t^2 - 4$, so the fibre $\psi^{-1}(t)$ lies on the conic
      $z^2 = t^2 - 4$ over $\Fp$.
\item The polynomial $t^3 - 4t$ factors as
\[
t^3 - 4t = t(t^2 - 4) = t \cdot z(y)^2,
\]
so $\chi(t^3 - 4t) = \chi(t)\cdot\chi(z(y))^2 = \chi(t)\cdot\chi(t^2-4)$.
\item Under the change of variables $x = t$, $w = t \cdot z(y)$, the
      elimination of the torus coordinate $y$ yields the elliptic curve
\[
E\colon w^2 = x^3 - 4x.
\]
In particular, $E$ arises canonically as the pushforward of the torus
$\Torus(\Fp)$ under $\psi$.
\end{enumerate}
\end{lemma}

\begin{proof}
\textbf{Part~(i).}
Since $y \in \Torus(\Fp)$ satisfies $y^{p+1} = 1$ (i.e., $y \cdot y^{-1} = 1$),
a direct computation gives
\[
z(y)^2 = (y - y^{-1})^2 = y^2 - 2 + y^{-2}
= (y + y^{-1})^2 - 4 = t^2 - 4.
\]

\textbf{Part~(ii).}
Substituting $t^2 - 4 = z(y)^2$:
\[
t^3 - 4t = t(t^2 - 4) = t \cdot z(y)^2.
\]
Since $\chi$ is completely multiplicative and $\chi(z(y)^2) = \chi(z(y))^2 = 1$
(a perfect square contributes $+1$ unless $z(y)=0$, handled separately),
we obtain
\[
\chi(t^3 - 4t) = \chi(t) \cdot \chi(t^2 - 4).
\]
The exceptional case $z(y) = 0$ (i.e., $y = \pm 1$, the fixed points of $\psi$)
contributes at most $O(1)$ terms and does not affect the sum.

\textbf{Part~(iii).}
Set $x = t = y + y^{-1}$ and $w = t \cdot z(y) = (y+y^{-1})(y-y^{-1})$.
Then
\[
w^2 = t^2 \cdot z(y)^2 = t^2(t^2-4) = t^4 - 4t^2.
\]
This is a genus-$1$ model in $(x,w)$. To obtain Weierstrass form, we pass
to the pushforward character sum: since every value $t \in \psi(\Torus)$ is
attained by a pair $\{y, y^{-1}\}$ with $z(y)^2 = t^2-4$, counting
$\mathbb{F}_p$-points weighted by $\chi$ gives
\[
\sum_{y \in \Torus(\Fp)} \chi(\psi(y))
= \sum_{t \in \Fp} \chi(t) \cdot \#\psi^{-1}(t)_{\mathrm{sgn}}
= \sum_{t \in \Fp} \chi(t^3 - 4t),
\]
where the last equality follows from the factorisation in~(ii) and the
identity $\sum_{y:\psi(y)=t}\chi(y-y^{-1}) = \chi(t^2-4)$
(each fibre contributes $\chi(z) + \chi(-z) = 2\chi(z)$ for $t\ne\pm 2$,
but the factor of $2$ cancels against the $\tfrac{1}{2}$ from the $2$-to-$1$
map). The right-hand side is exactly $-a_p(E)$ for $E\colon w^2 = x^3-4x$
by the Hasse--Weil point count \cite{Silverman2009}.
\end{proof}

\begin{remark}[CM Structure of $E$]
\label{rem:cm-structure}
The elliptic curve $E\colon y^2 = x^3 - 4x$ has $j$-invariant $1728$ and admits
complex multiplication by $\mathbb{Z}[i]$ \cite{Silverman1994}.
This CM structure is a \emph{consequence} of the torus geometry:
the factorisation $x^3 - 4x = x(x-2)(x+2)$ reflects the three fixed points of
the trace morphism ($t = 0, \pm 2$, corresponding to $y = \pm i, \pm 1$),
and the symmetry $x \mapsto -x$ of $E$ corresponds to the involution
$y \mapsto -y$ on $\Torus(\Fp)$.
The CM field $\mathbb{Q}(i)$ governs the split/inert dichotomy of $a_p(E)$
(see \cref{rem:two-inert}), independently of the Fibonacci inertness condition
$\bigl(\tfrac{5}{p}\bigr) = -1$.
\end{remark}

\begin{lemma}[Elliptic Reduction]
\label{lem:elliptic}
Let $E/\mathbb{F}_p\colon y^2 = x^3 - 4x$. Then
\[
S_T := \sum_{t \in \Fp} \chi(t^3 - 4t) = -a_p(E),
\]
and the density of QNRs in the trace image satisfies
\[
\delta_p = \frac{1}{2} - \frac{a_p(E)}{2(p+1)} + O\!\left(\frac{1}{p}\right).
\]
\end{lemma}

\begin{proof}
The identity $S_T = -a_p(E)$ is immediate from \cref{lem:cubic-origin}(iii)
and the standard Hasse--Weil point count:
\[
a_p(E) = p + 1 - \#E(\mathbb{F}_p)
= p + 1 - \left(p + 1 + \sum_{t \in \Fp} \chi(t^3-4t)\right)
= -S_T.
\]
Here $\#E(\mathbb{F}_p) = \sum_{t \in \Fp}(1 + \chi(t^3-4t)) + 1$,
where $1 + \chi(t^3-4t)$ counts $y$ with $y^2 = t^3-4t$
(giving $2, 1, 0$ for non-zero QR, zero, QNR respectively)
and the final $+1$ is the point at infinity \cite{Silverman2009}.

The density formula follows from the $2$-to-$1$ folding
$\psi\colon \Torus(\Fp) \to \Fp$ over $(p+3)/2$ values
(\cref{cor:trace-cardinality}): among these values, the fraction
that are QNR equals $\tfrac{1}{2} - S_T/(2(p+1)) + O(1/p)$.
\end{proof}

\begin{corollary}[Square-Root Decay]
\label{cor:sqrt-decay}
\[
\left|\delta_p - \frac{1}{2}\right|
= \frac{|a_p(E)|}{2(p+1)} \le \frac{2\sqrt{p}}{2(p+1)} = O(p^{-1/2}),
\]
so $\delta_p \to 1/2$ as $p \to \infty$.
\end{corollary}

\begin{proof}
Immediate from \cref{lem:elliptic} and the Hasse bound $|a_p(E)|\le 2\sqrt{p}$
\cite{Silverman2009}.
\end{proof}

% ============================================================================
\section{Proof of the Main Theorem}
\label{sec:main-proof}
% ============================================================================

We now assemble the preceding results to prove \cref{thm:main-formal}.

\begin{proof}[Proof of Theorem~\ref{thm:main-formal}]
\textbf{Step 1: Algebraic reformulation.}
Since $\pi(p) = p+1$ for inert primes \cite{Wall1960}, the character sum
$S_p$ runs over one full Pisano period.
Applying \cref{prop:binet-torus} and the $2$-to-$1$ folding of
\cref{lem:decomposition}, the character sum over Fibonacci indices
reduces to a character sum over the torus subgroup
$H_p = \langle \alpha \rangle \le \Torus(\Fp)$.

\medskip
\noindent\textbf{Step 2: Fourier decomposition.}
By \cref{lem:decomposition},
\[
S_p = \frac{1}{d} \sum_{\substack{\omega^d = 1}} T(\omega),
\qquad
T(\omega) := \sum_{y \in \Torus(\Fp)} \omega(y)\,\chi(\psi(y)).
\]
The trivial character $\omega = \mathbf{1}$ contributes
$T(\mathbf{1}) = \sum_{y \in \Torus} \chi(\psi(y))$, which by
\cref{lem:elliptic} equals $-a_p(E) = O(\sqrt{p})$.

\medskip
\noindent\textbf{Step 3: Non-trivial character bound.}
For each non-trivial $\omega$ with $\omega^d = 1$, \cref{prop:non-degenerate}
guarantees that $T(\omega)$ is non-constant, so \cref{lem:hybrid-bound} applies
and gives $|T(\omega)| \le 2\sqrt{p}$.

\medskip
\noindent\textbf{Step 4: Conclusion.}
Since $\Torus(\Fp)$ is cyclic of order $p+1$ (see \cref{prop:dual-torus}(i)),
and $d \mid (p+1)$, the number of characters $\omega \in \widehat{\Torus(\Fp)}$
satisfying $\omega^d = 1$ is exactly $d$.
Combining Steps 2--3,
\[
|S_p| \le \frac{1}{d}\sum_{\omega^d=1}|T(\omega)|
\le \frac{d \cdot 2\sqrt{p}}{d} = 2\sqrt{p}.
\]
This proves part~(i) with constant $C=2$.

Part~(ii) follows immediately: since $\pi(p) = p+1$,
\[
\delta_p = \frac{1}{2} - \frac{S_p}{2(p+1)} = \frac{1}{2} + O(p^{-1/2}).
\]
The rate $O(p^{-1/2})$ is made explicit in \cref{cor:sqrt-decay}.

Part~(iii) is \cref{lem:elliptic}.
\end{proof}

% ============================================================================
\section{Conclusion and Further Directions}
\label{sec:conclusion}
% ============================================================================

\subsection{Summary}

This paper has established a precise arithmetic bridge between the quadratic
residuosity pattern of Fibonacci sequences and the geometry of a CM elliptic
curve. The central identity
\[
S_p \;=\; \sum_{n=1}^{p+1} \chi(F_n \bmod p) \;=\; -a_p(E),
\qquad E\colon y^2 = x^3 - 4x,
\]
proved in \cref{thm:main-formal}, shows that the apparent $3/8$ bias observed
for small inert primes is not a property of the Fibonacci dynamics itself, but
rather a transient geometric artifact of the trace morphism's image structure
on the norm-one torus $\Torus(\Fp)$. The equidistribution rate
$\delta_p = \tfrac{1}{2} + O(p^{-1/2})$ is governed entirely by the Hasse
bound on $a_p(E)$, with the $O(\sqrt{p})$ constant explicit and equal to $2$.

The quadratic residuosity pattern in Fibonacci sequences is governed by the
Frobenius action on a CM elliptic curve, revealing a structural bridge between
toric dynamics and elliptic cohomology. This places the Fibonacci bias firmly
within the framework of $\ell$-adic geometry, rather than probabilistic
heuristics.

\subsection{Comparison with Prior Work}

While Shparlinski \cite{Shparlinski2009} obtained distribution results for
Fibonacci numbers modulo primes via analytic exponential sum techniques, the
present work provides an \emph{exact} identity linking the quadratic character
sum to the Frobenius trace of a specific CM elliptic curve. This exactness
yields a sharper geometric explanation: the bias is not merely bounded, but
is determined by a single arithmetic invariant $a_p(E)$ whose distribution
follows the CM Sato--Tate law (\cref{sec:numerics}). The methods of
Cambrea--Javaheri \cite{CambreaJavaheri2022} and Javaheri--Krylov
\cite{JavaheriKrylov2020} address related combinatorial questions but do not
establish this elliptic correspondence.

\subsection{Open Problems}

The results of this paper suggest several natural directions for further
investigation.

\begin{enumerate}[(1)]

\item \textbf{Higher-order residue characters.}
Does an analogue of \cref{thm:main-formal} hold for cubic or quartic residue
characters? Specifically, for the cubic character $\chi_3$ on $\mathbb{F}_p$
(with $p \equiv 1 \pmod{3}$), does
\[
\sum_{n=1}^{\pi(p)} \chi_3(F_n)
\]
admit an exact expression in terms of Hecke Gr\"{o}ssencharacters or
Jacobi sums of a higher-genus curve?

\item \textbf{Lucas sequences and generalised recurrences.}
The methods developed here extend naturally to Lucas sequences
$V_n = \alpha^n + \alpha^{-n}$ and more general second-order recurrences
$U_{n+1} = aU_n + bU_{n-1}$. The torus $T(\mathbb{F}_p)$ is replaced by
a twist, and the elliptic correspondence may yield a different CM curve
depending on the discriminant $a^2 - 4b$.

\item \textbf{Non-CM analogues.}
For a prime $p$ such that $\alpha$ generates a subgroup $H_p$ of large index
$d$ in $\Torus(\Fp)$, the character sum $S_p$ is expressed as a combination
of $d$ hybrid sums (see Appendix~B). Can one identify the corresponding
arithmetic object (motif, $L$-function) in the non-CM setting, where the
Sato--Tate distribution is semicircular rather than arcsine?

\item \textbf{Effective lower bound on $\operatorname{ord}(\alpha)$.}
The equidistribution rate in \cref{thm:main-formal}(ii) can be sharpened
from $O(p^{-1/2})$ to $O(d/p)$ if one establishes
$\operatorname{ord}(\alpha) \gg p^{1/2}$ for a density-one set of inert
primes (see \cref{rem:artin-grh}). This is an instance of the multiplicative
order problem on algebraic tori; conditional results under the Generalised
Riemann Hypothesis are expected from standard methods in Galois representations
\cite{Serre1968}, but an unconditional proof remains open.

\item \textbf{The split prime case $\bigl(\tfrac{5}{p}\bigr) = +1$.}
When $p$ splits in $\mathbb{Q}(\sqrt{5})$, both $\alpha$ and $\beta$ lie in
$\mathbb{F}_p^\times$ and the Fibonacci orbit is confined to a subgroup of
$\mathbb{F}_p^\times$ rather than the norm-one torus. The character sum
$S_p$ decomposes differently, and while square-root cancellation still holds
via standard character sum bounds \cite{IrelandRosen1990}, the exact
elliptic identity of \cref{thm:main-formal}(iii) does not have a direct
analogue. Determining the precise equidistribution structure in this case
is a natural companion problem.

\end{enumerate}

\begin{remark}[Arithmetic Significance]
The identity $S_p = -a_p(E)$ suggests that the full $L$-function of
$E/\mathbb{Q}$ encodes the asymptotics of the Fibonacci quadratic bias across
all inert primes simultaneously. A proof of the Birch--Swinnerton-Dyer
conjecture for $E$ would in principle yield precise information about the
distribution of $\{S_p\}_{p \text{ inert}}$ via the functional equation of
$L(E,s)$.
\end{remark}


% ============================================================================

We verify our theoretical predictions by high-performance computation on the CM
elliptic curve $E\colon y^2 = x^3 - 4x$ for all $148{,}932$ primes
$3 \le p \le 1{,}999{,}993$.

\begin{remark}[Two notions of inertness]
\label{rem:two-inert}
Throughout this section, \emph{split} and \emph{inert} refer to the behaviour of $p$
in the CM field $\mathbb{Q}(i)$ of the curve $E$, governed by the congruence
condition modulo $4$:
\[
p \equiv 1 \pmod{4} \;\text{(split for } E\text{)},
\qquad
p \equiv 3 \pmod{4} \;\text{(inert for } E\text{)}.
\]
This is distinct from the inertness condition $\bigl(\tfrac{5}{p}\bigr) = -1$
(i.e., $p \equiv \pm 2 \pmod{5}$) used in the main theorem to ensure that
$\alpha \in \mathbb{F}_{p^2} \setminus \mathbb{F}_p$.
For the numerical experiments, all primes satisfy both conditions simultaneously,
since we restrict to primes $p > 5$ with $\bigl(\tfrac{5}{p}\bigr) = -1$.
The CM dichotomy $p \bmod 4$ then governs whether $a_p(E)$ vanishes or not.
\end{remark}

\subsection{Normalized Frobenius Traces}

Figure~\ref{fig:trace} confirms the Weil bound and the CM split/inert dichotomy
for $E\colon y^2 = x^3 - 4x$ (see \cref{rem:two-inert}).
Split primes ($p \equiv 1 \pmod{4}$) exhibit traces distributed across
$[-2\sqrt{p},\,2\sqrt{p}]$, while inert primes ($p \equiv 3 \pmod{4}$)
satisfy $a_p = 0$ exactly, as predicted by the CM theory \cite{Silverman1994}.

\begin{figure}[H]
\centering
\includegraphics[width=0.85\textwidth]{Fig1_Trace_Analysis.png}
\caption{Normalized Frobenius traces $a_p/\sqrt{p}$ for $E\colon y^2 = x^3 - 4x$
verify the Weil bound and split/inert dichotomy.}
\label{fig:trace}
\end{figure}

\subsection{CM Sato--Tate Distribution}

Figure~\ref{fig:sato-tate} illustrates the CM Sato--Tate density for split
primes \cite{Silverman1994,Serre1968}. The empirical histogram matches the
theoretical U-shaped distribution $(\pi\sqrt{4-x^2})^{-1}$.

\begin{figure}[H]
\centering
\includegraphics[width=0.85\textwidth]{Fig2_SatoTate.png}
\caption{Trace distribution for split primes matches the CM Sato--Tate
theoretical density (dashed curve) \cite{Silverman1994}.}
\label{fig:sato-tate}
\end{figure}

\subsection{Chebotarev Density}

Figure~\ref{fig:chebotarev} verifies the Chebotarev density prediction
\cite{DiamondShurman2005}. The cumulative ratio of inert primes converges to
$1/2$ as predicted by class field theory.

\begin{figure}[H]
\centering
\includegraphics[width=0.85\textwidth]{Fig3_Convergence.png}
\caption{Convergence to Chebotarev density: cumulative ratio of inert primes
approaches $1/2$ \cite{DiamondShurman2005}.}
\label{fig:chebotarev}
\end{figure}

\FloatBarrier
\subsection{Computational Results}

For all primes $3 \le p \le 1{,}999{,}993$:
\begin{itemize}
\item Total primes computed: $148{,}932$
\item Split for $E$ ($p \equiv 1 \pmod{4}$): $74{,}416$
\item Inert for $E$ ($p \equiv 3 \pmod{4}$): $74{,}516$; \quad $a_p = 0$ verified exactly for all
\item Empirical inert ratio: $0.500336$ \quad (Chebotarev prediction: $0.500000$)
\item Maximum observed Weil ratio: $|a_p|/(2\sqrt{p}) = 0.999999 < 1$ (bound never attained)
\item Maximum Pisano period: $\pi(p) = 3{,}999{,}988$
\item Computation time: $\approx 11$ minutes on 15-core hardware (Numba JIT, multiprocessing)
\end{itemize}

\subsection{Worked Example: The Case \texorpdfstring{$p=7$}{p=7}}
\label{subsec:worked-example}

We illustrate the main identity $S_p = -a_p(E)$ concretely for $p = 7$.
Since $\bigl(\tfrac{5}{7}\bigr) = \bigl(\tfrac{5}{7}\bigr)$: as $5 \equiv 5
\pmod{7}$ and $5^{(7-1)/2} = 5^3 = 125 \equiv 6 \equiv -1 \pmod{7}$, we have
$\bigl(\tfrac{5}{7}\bigr) = -1$, confirming $p=7$ is inert. The Pisano period
is $\pi(7) = 8 = 7+1$, consistent with \cref{thm:main-formal}.

\medskip
\noindent\textbf{Step 1: Fibonacci sequence mod 7.}

\begin{table}[ht]
\centering
\begin{tabular}{@{}ccccccccc@{}}
\toprule
$n$ & $1$ & $2$ & $3$ & $4$ & $5$ & $6$ & $7$ & $8$ \\
\midrule
$F_n \bmod 7$ & $1$ & $1$ & $2$ & $3$ & $5$ & $1$ & $6$ & $0$ \\
$\chi(F_n)$   & $1$ & $1$ & $1$ & $1$ & $-1$ & $1$ & $-1$ & $0$ \\
\bottomrule
\end{tabular}
\caption{Fibonacci residues and Legendre symbols modulo $7$.
QRs mod $7$: $\{1,2,4\}$; QNRs: $\{3,5,6\}$.
Note $F_8 = 0$ contributes $\chi(0)=0$.}
\label{tab:p7-example}
\end{table}

\noindent Here $1^2\equiv 1$, $2^2\equiv 4$, $3^2\equiv 2\pmod 7$, so
$\mathrm{QR}_7 = \{1,2,4\}$ and $\mathrm{QNR}_7 = \{3,5,6\}$.

\medskip
\noindent\textbf{Step 2: Character sum.}
\[
S_7 = \chi(1)+\chi(1)+\chi(2)+\chi(3)+\chi(5)+\chi(1)+\chi(6)+\chi(0)
= 1+1+1+(-1)+(-1)+1+(-1)+0 = 1.
\]

\medskip
\noindent\textbf{Step 3: Point count on $E\colon y^2 = x^3 - 4x$ over $\mathbb{F}_7$.}

\begin{table}[ht]
\centering
\begin{tabular}{@{}ccccc@{}}
\toprule
$x$ & $x^3-4x \bmod 7$ & $\chi(x^3-4x)$ & \# points $(x,y)$ \\
\midrule
$0$ & $0$  & $0$  & $1$ \\
$1$ & $4$  & $1$  & $2$ \\
$2$ & $0$  & $0$  & $1$ \\
$3$ & $1$  & $1$  & $2$ \\
$4$ & $6$  & $-1$ & $0$ \\
$5$ & $2$  & $1$  & $2$ \\
$6$ & $4$  & $1$  & $2$ \\
\bottomrule
\end{tabular}
\caption{Point count on $E$ over $\mathbb{F}_7$ (affine points only).}
\label{tab:p7-elliptic}
\end{table}

\noindent Affine points: $1+2+1+2+0+2+2 = 10$. Adding the point at infinity:
$\#E(\mathbb{F}_7) = 11$.

\medskip
\noindent\textbf{Step 4: Frobenius trace.}
\[
a_7(E) = 7 + 1 - \#E(\mathbb{F}_7) = 8 - 11 = -3.
\]

\medskip
\noindent\textbf{Step 5: Verification of the main identity.}
From \cref{thm:main-formal}(iii),
\[
\sum_{t \in \mathbb{F}_7} \chi(t^3-4t)
= 1+1+(-1)+1+1+0+0 = \ldots
\]
Direct computation: $\sum_{t=0}^{6}\chi(t^3-4t) = 0+1+0+1+(-1)+1+1 = 3 = -a_7(E)$. \checkmark

\medskip
\noindent The QNR density is $\delta_7 = 2/8 = 1/4$, while the theoretical
prediction gives $\tfrac{1}{2} - a_7(E)/(2\cdot 8) = \tfrac{1}{2}+\tfrac{3}{16}
= \tfrac{11}{16}$. The discrepancy $|1/4 - 1/2| = 1/4 \le 2\sqrt{7}/(2\cdot 8)
\approx 0.331$ is consistent with \cref{cor:sqrt-decay}.

\FloatBarrier

\appendix

\section{Computational Note}
\label{app:computational}

The numerical experiments in \cref{sec:numerics} used an optimized Python
implementation of the character sum
\[
S_p = \sum_{t \in \mathbb{F}_p} \chi(t^3 - 4t) = -a_p(E),
\]
for the CM elliptic curve $E\colon y^2 = x^3 - 4x$. All computations were
verified for all $148{,}932$ primes up to $p = 1{,}999{,}993$ using Numba JIT
compilation and multiprocessing parallelisation across 15 cores ($\approx 11$~min).
The full source code and dataset (CSV and Excel format) are
provided in the Supplementary Material.

\section{Restricted Character Spectrum}
\label{app:restricted-spectrum}

Throughout this appendix, the symbol $\psi$ denotes the trace morphism
$y \mapsto y + y^{-1}$. Additive characters on $\mathbb{F}_p$ are denoted
by $\psi_{\mathrm{add}}$, as specified in the notation table.

\begin{lemma}[Restricted Character Spectrum]
\label{lem:restricted-spectrum}
Let $\Torus(\Fp) = \langle \alpha \rangle$ be cyclic of order $p+1$,
and let $H = \langle \alpha \rangle$ be a subgroup of index
$k = [\Torus(\Fp) : H] \ge 1$.
Let $\psi_{\mathrm{add}}\colon \Fp \to \mathbb{C}^\times$ be a non-trivial
additive character of $\mathbb{F}_p$.
The character sum
\[
S(\psi_{\mathrm{add}}) := \sum_{x \in H}
\psi_{\mathrm{add}}\!\bigl(\mathrm{Tr}_{\mathbb{F}_{p^2}/\Fp}(x)\bigr)
\]
admits the decomposition
\[
S(\psi_{\mathrm{add}})
= \frac{1}{k}
\sum_{\omega \in H^\perp}
\sum_{y \in \Torus(\Fp)}
\omega(y)\,\psi_{\mathrm{add}}(\psi(y)),
\]
where $H^\perp \subseteq \widehat{\Torus(\Fp)}$ is the annihilator of $H$
(consisting of exactly $k$ multiplicative characters),
and $\psi(y) = y + y^{-1}$ is the trace morphism.
In particular, the Fourier expansion of the trace distribution
involves at most $k$ multiplicative characters of $\Torus(\Fp)$.
\end{lemma}

\begin{proof}
Since $\Torus(\Fp)$ is cyclic of order $p+1$ and $H$ has index $k$,
character orthogonality on $\Torus(\Fp)$ gives
\[
\mathbf{1}_H(y)
= \frac{1}{k}\sum_{\omega \in H^\perp} \omega(y),
\]
where $|H^\perp| = k$ (see \cref{prop:dual-torus}(iii)).
Substituting and using $\mathrm{Tr}(y) = \psi(y) = y + y^{-1}$
for $y \in \Torus(\Fp)$ (see \cref{lem:frobenius-action}):
\[
S(\psi_{\mathrm{add}})
= \sum_{y \in \Torus(\Fp)} \mathbf{1}_H(y)\,\psi_{\mathrm{add}}(\psi(y))
= \frac{1}{k}\sum_{\omega \in H^\perp}
\underbrace{\sum_{y \in \Torus(\Fp)} \omega(y)\,\psi_{\mathrm{add}}(\psi(y))}_{T(\omega,\,\psi_{\mathrm{add}})}.
\]
This is the Fourier decomposition over $H^\perp$,
confirming that $S(\psi_{\mathrm{add}})$ is a linear combination of exactly $k$
twisted sums $T(\omega, \psi_{\mathrm{add}})$.
\end{proof}

\begin{remark}[Compatibility with Square-Root Cancellation]
\label{rem:spectrum-compatibility}
The restriction to at most $k$ multiplicative characters does not contradict
the square-root bounds established in \cref{prop:non-degenerate,lem:hybrid-bound}.
Each twisted sum satisfies
\[
\bigl|T(\omega,\psi_{\mathrm{add}})\bigr|
= \left|\sum_{y \in \Torus(\Fp)} \omega(y)\,\psi_{\mathrm{add}}(\psi(y))\right|
\le 2\sqrt{p}
\]
by \cref{lem:hybrid-bound} (Deligne's Riemann Hypothesis for the
sheaf $\mathcal{L}_\omega \otimes \psi^*\mathcal{L}_{\psi_{\mathrm{add}}}$).
Hence, asymptotic equidistribution holds regardless of $k$:
\[
|S(\psi_{\mathrm{add}})| \le \frac{1}{k}\cdot k \cdot 2\sqrt{p} = 2\sqrt{p}.
\]
However, when $k > 1$, the trace distribution is expressed as a linear
combination of only $k$ twisted sums rather than a full Fourier expansion.
This restricted spectral support explains why exact cancellation
($S(\psi_{\mathrm{add}}) = 0$) is not generic when the subgroup index exceeds $1$:
each $T(\omega, \psi_{\mathrm{add}})$ individually attains values spread across
$[-2\sqrt{p}, 2\sqrt{p}]$, and the partial cancellation between the $k$ terms
depends on the arithmetic of $\alpha$ modulo $p$.
\end{remark}

\section*{Acknowledgments}
The author thanks the anonymous referees for their careful reading and
valuable suggestions, which improved the clarity and rigour of this paper.

\section*{Conflict of Interest}
The author declares no conflict of interest.

\section*{Data Availability Statement}
The computational data and source code supporting the findings of this
study are provided in the Supplementary Material (Python source, CSV
dataset, and Excel summary). The code is also publicly available at
\url{https://github.com/majidghandali/fibonacci-cm-elliptic}.

\bibliographystyle{unsrtnat}
\bibliography{references}

\end{document}
