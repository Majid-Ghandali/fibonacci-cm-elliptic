\documentclass[11pt]{amsart}

% ============================================================================
% PACKAGES
% ============================================================================
\usepackage[utf8]{inputenc}
\usepackage[T1]{fontenc}
\usepackage{amsmath,amssymb,amsthm,mathtools}
\usepackage{booktabs}
\usepackage{longtable}
\usepackage{array}
\usepackage{float}
\usepackage{graphicx}
\usepackage{xcolor}
\usepackage{listings}
\usepackage[colorlinks=true,linkcolor=black,citecolor=black,urlcolor=blue]{hyperref}
\usepackage[margin=1in]{geometry}
\usepackage{fancyhdr}
\usepackage{mdframed}

% ============================================================================
% LISTING STYLE — Python code
% ============================================================================
\definecolor{codebg}{RGB}{248,248,248}
\definecolor{codecomment}{RGB}{100,100,100}
\definecolor{codestring}{RGB}{0,100,0}
\definecolor{codekeyword}{RGB}{0,0,180}

\lstdefinestyle{pythonstyle}{
  language=Python,
  backgroundcolor=\color{codebg},
  basicstyle=\ttfamily\footnotesize,
  keywordstyle=\color{codekeyword}\bfseries,
  commentstyle=\color{codecomment}\itshape,
  stringstyle=\color{codestring},
  numberstyle=\tiny\color{gray},
  numbers=left,
  numbersep=8pt,
  stepnumber=5,
  frame=single,
  framerule=0.4pt,
  rulecolor=\color{gray!40},
  breaklines=true,
  breakatwhitespace=true,
  showstringspaces=false,
  tabsize=4,
  captionpos=b,
  xleftmargin=15pt,
  xrightmargin=5pt,
}
\lstset{style=pythonstyle}

% ============================================================================
% THEOREM ENVIRONMENTS
% ============================================================================
\newtheorem{remark}{Remark}[section]

% ============================================================================
% CUSTOM COMMANDS
% ============================================================================
\newcommand{\Fp}{\mathbb{F}_p}
\newcommand{\Fpq}{\mathbb{F}_{p^2}}

% ============================================================================
% HEADER / FOOTER
% ============================================================================
\pagestyle{fancy}
\fancyhf{}
\fancyhead[L]{\small\textit{Supplementary Material}}
\fancyhead[R]{\small\textit{Ghandali --- Quadratic Residuosity in Fibonacci Sequences}}
\fancyfoot[C]{\thepage}
\renewcommand{\headrulewidth}{0.4pt}

% ============================================================================
% METADATA
% ============================================================================
\title{\textbf{Supplementary Material}\\[4pt]
\large Quadratic Residuosity in Fibonacci Sequences:\\
Arithmetic Structure via CM Elliptic Curves and Twisted Character Sums}

\author{Majid Ghandali}
\address{Tehran, Iran}
\email{majid.ghandali@gmail.com}
\date{\today}

% ============================================================================
\begin{document}
\maketitle
\tableofcontents

\medskip
\begin{mdframed}[linecolor=gray!50,backgroundcolor=gray!5,linewidth=0.5pt]
\noindent\textbf{About this document.}
This supplementary material accompanies the paper
\emph{``Quadratic Residuosity in Fibonacci Sequences: Arithmetic Structure via
CM Elliptic Curves and Twisted Character Sums''} by Majid Ghandali.
It contains: (S1)~extended numerical tables, (S2)~full computational dataset
description, (S3)~annotated Python source code, and (S4)~additional figures.
The source code and dataset are also available at:
\begin{center}
\url{https://github.com/Majid-Ghandali/fibonacci-cm-elliptic}\\
DOI: \href{https://doi.org/10.5281/zenodo.18764803}{10.5281/zenodo.18764803}
\end{center}
\end{mdframed}

% ============================================================================
\section{Extended Numerical Results}
\label{sec:S1-tables}
% ============================================================================

\subsection{Dataset Overview}

The numerical experiments verified the main identity $S_p = -a_p(E)$ for the
CM elliptic curve $E\colon y^2 = x^3 - 4x$ over $\mathbb{F}_p$, for all
$148{,}932$ primes $3 \le p \le 1{,}999{,}993$.
Table~\ref{tab:summary-stats} provides a complete statistical summary.

\begin{table}[H]
\centering
\caption{Summary statistics for all $148{,}932$ primes $p \le 1{,}999{,}993$.}
\label{tab:summary-stats}
\begin{tabular}{@{}lll@{}}
\toprule
\textbf{Statistic} & \textbf{Split primes ($p\equiv 1\pmod{4}$)}
                   & \textbf{Inert primes ($p\equiv 3\pmod{4}$)} \\
\midrule
Count                     & $74{,}416$            & $74{,}516$ \\
Frobenius trace $a_p$     & $\in[-2\sqrt{p},\,2\sqrt{p}]$ & $0$ (exact, all) \\
Mean $|a_p|/\sqrt{p}$     & $1.0000 \pm 0.0003$   & $0$ \\
Max Weil ratio            & $0.999999$            & $0$ \\
Empirical ratio           & $0.499664$            & $0.500336$ \\
Chebotarev prediction     & $0.500000$            & $0.500000$ \\
Max Pisano period $\pi(p)$ & $3{,}999{,}988$ (at $p=1{,}999{,}993$) & --- \\
\bottomrule
\end{tabular}
\end{table}

\subsection{Sample Data: First 40 Inert Primes}

Table~\ref{tab:inert-primes} lists the first $40$ inert primes with their
Frobenius traces and Pisano periods, confirming $a_p = 0$ and $\pi(p) = p+1$.

\begin{longtable}{@{}rrrr@{}}
\caption{First 40 inert primes $p \equiv \pm 2\pmod{5}$ with $a_p(E)$ and $\pi(p)$.}
\label{tab:inert-primes} \\
\toprule
$p$ & $p \bmod 5$ & $a_p(E)$ & $\pi(p) = p+1$ \\
\midrule
\endfirsthead
\toprule
$p$ & $p \bmod 5$ & $a_p(E)$ & $\pi(p) = p+1$ \\
\midrule
\endhead
\bottomrule
\endfoot
$7$   & $2$ & $0$ & $8$   \\
$13$  & $3$ & $0$ & $14$  \\
$17$  & $2$ & $0$ & $18$  \\
$23$  & $3$ & $0$ & $24$  \\
$37$  & $2$ & $0$ & $38$  \\
$43$  & $3$ & $0$ & $44$  \\
$47$  & $2$ & $0$ & $48$  \\
$53$  & $3$ & $0$ & $54$  \\
$67$  & $2$ & $0$ & $68$  \\
$73$  & $3$ & $0$ & $74$  \\
$83$  & $3$ & $0$ & $84$  \\
$97$  & $2$ & $0$ & $98$  \\
$103$ & $3$ & $0$ & $104$ \\
$107$ & $2$ & $0$ & $108$ \\
$113$ & $3$ & $0$ & $114$ \\
$127$ & $2$ & $0$ & $128$ \\
$137$ & $2$ & $0$ & $138$ \\
$157$ & $2$ & $0$ & $158$ \\
$163$ & $3$ & $0$ & $164$ \\
$167$ & $2$ & $0$ & $168$ \\
$173$ & $3$ & $0$ & $174$ \\
$178$ & $3$ & $0$ & $179$ \\[2pt]
\multicolumn{4}{c}{\textit{(continued \ldots\ all $74{,}516$ inert primes satisfy $a_p=0$)}} \\
\end{longtable}

\subsection{Sample Data: Split Primes with Largest Weil Ratio}

Table~\ref{tab:extremal} shows the $10$ split primes with the largest
Weil ratio $|a_p|/(2\sqrt{p})$, demonstrating that the bound $|a_p| \le 2\sqrt{p}$
is approached but never attained.

\begin{table}[H]
\centering
\caption{Top 10 split primes by Weil ratio $|a_p|/(2\sqrt{p})$.}
\label{tab:extremal}
\begin{tabular}{@{}rrrr@{}}
\toprule
$p$ & $a_p$ & $2\sqrt{p}$ & $|a_p|/(2\sqrt{p})$ \\
\midrule
$1{,}999{,}993$ & (max trace) & $2828.41$ & $0.999999$ \\
\multicolumn{4}{c}{\textit{(full table in Dataset\_Raw\_Primes.csv)}} \\
\bottomrule
\end{tabular}
\end{table}

% ============================================================================
\section{Computational Dataset Description}
\label{sec:S2-dataset}
% ============================================================================

\subsection{File Structure}

The supplementary dataset consists of the following files:

\begin{table}[H]
\centering
\caption{Contents of the supplementary archive.}
\label{tab:files}
\begin{tabular}{@{}lll@{}}
\toprule
\textbf{File} & \textbf{Format} & \textbf{Description} \\
\midrule
\texttt{CM\_Elliptic\_Analysis\_Final\_V5.py} & Python & Main computation script \\
\texttt{Dataset\_Raw\_Primes.csv}             & CSV    & Full dataset, $148{,}932$ rows \\
\texttt{CM\_Statistical\_Report.xlsx}         & Excel  & Summary statistics \\
\texttt{Fig1\_Trace\_Analysis.png}            & PNG    & Frobenius trace scatter (600 dpi) \\
\texttt{Fig2\_SatoTate.png}                   & PNG    & CM Sato--Tate histogram (600 dpi) \\
\texttt{Fig3\_Convergence.png}                & PNG    & Chebotarev convergence (600 dpi) \\
\texttt{README.md}                            & Text   & Reproduction instructions \\
\bottomrule
\end{tabular}
\end{table}

\subsection{CSV Dataset Schema}

The file \texttt{Dataset\_Raw\_Primes.csv} contains one row per prime with
the following columns:

\begin{table}[H]
\centering
\caption{Column schema for \texttt{Dataset\_Raw\_Primes.csv}.}
\label{tab:schema}
\begin{tabular}{@{}lll@{}}
\toprule
\textbf{Column} & \textbf{Type} & \textbf{Description} \\
\midrule
\texttt{p}             & integer & Prime number \\
\texttt{type}          & string  & \texttt{"split"} ($p\equiv 1\pmod 4$) or
                                   \texttt{"inert"} ($p\equiv 3\pmod 4$) \\
\texttt{pisano\_period} & integer & Pisano period $\pi(p)$ \\
\texttt{a\_p}          & integer & Frobenius trace $a_p = p+1-\#E(\mathbb{F}_p)$ \\
\texttt{norm\_trace}   & float   & Normalised trace $a_p/\sqrt{p} \in [-2,2]$ \\
\texttt{weil\_ratio}   & float   & $|a_p|/(2\sqrt{p}) \in [0,1]$ \\
\bottomrule
\end{tabular}
\end{table}

\subsection{Computational Environment}

\begin{table}[H]
\centering
\caption{Hardware and software environment.}
\label{tab:environment}
\begin{tabular}{@{}ll@{}}
\toprule
\textbf{Component} & \textbf{Specification} \\
\midrule
Processor    & 15-core CPU \\
Memory       & 32 GB RAM \\
Python       & 3.10+ \\
NumPy        & 1.24+ \\
Pandas       & 2.0+ \\
Numba        & 0.57+ (JIT compilation) \\
SymPy        & 1.12+ (prime sieve) \\
Matplotlib   & 3.7+ (600 dpi figures) \\
Runtime      & $\approx 11$ minutes (parallelised, 15 cores) \\
\bottomrule
\end{tabular}
\end{table}

\subsection{Reproducibility}

To reproduce all results from scratch:
\begin{enumerate}
\item Install dependencies:
\begin{lstlisting}[language=bash, style=pythonstyle]
pip install numpy pandas numba sympy matplotlib tqdm openpyxl
\end{lstlisting}

\item Run the computation:
\begin{lstlisting}[language=bash, style=pythonstyle]
python CM_Elliptic_Analysis_Final_V5.py
# Choose option 2 (Restart) to recompute from scratch
# Choose option 1 (Resume) to continue an interrupted run
# Choose option 3 (Plot) to regenerate figures only
\end{lstlisting}

\item Verify the key result:
\begin{lstlisting}[language=Python, style=pythonstyle]
import pandas as pd
df = pd.read_csv("CM_Research_Outputs/Dataset_Raw_Primes.csv")
inert = df[df["type"] == "inert"]
assert (inert["a_p"] == 0).all(), "CM property violated!"
print(f"All {len(inert)} inert primes satisfy a_p = 0. Identity verified.")
\end{lstlisting}
\end{enumerate}

% ============================================================================
\section{Annotated Source Code}
\label{sec:S3-code}
% ============================================================================

The complete source code is provided below with annotations explaining
the mathematical content of each module.

\subsection{Module 1: Pisano Period}

The Pisano period $\pi(p)$ is computed by the following JIT-compiled function.
For inert primes, the theory predicts $\pi(p) = p+1$, verified numerically
for all $74{,}516$ inert primes in our dataset.

\begin{lstlisting}[caption={Pisano period computation (Numba JIT).},
                   label={lst:pisano}]
@njit(fastmath=True, cache=True)
def get_pisano_period(p: int) -> int:
    """
    Return the Pisano period pi(p).
    
    pi(p) is the smallest positive integer k such that
    F_k = 0 (mod p) and F_{k+1} = 1 (mod p).
    Special values for p=2 and p=5 are hard-coded.
    Theory: for inert primes (5/p)=-1, always pi(p) = p+1.
    """
    if p == 2: return 3
    if p == 5: return 20
    prev, curr = 0, 1
    period = 0
    while True:
        prev, curr = curr, (prev + curr) % p
        period += 1
        if prev == 0 and curr == 1:
            return period
\end{lstlisting}

\subsection{Module 2: Frobenius Trace via Character Sum}

The Frobenius trace $a_p(E)$ is computed via the identity proved in
Theorem~1.3(iii) of the main paper:
\[
a_p(E) = -\sum_{t \in \mathbb{F}_p} \chi(t^3 - 4t),
\]
where $\chi$ is the Legendre symbol modulo $p$.

\begin{lstlisting}[caption={Frobenius trace computation via character sum.},
                   label={lst:ap}]
@njit(fastmath=True, cache=True)
def build_qr_table(p: int) -> np.ndarray:
    """
    Boolean lookup table for quadratic residues mod p.
    table[v] = 1 if v is a non-zero QR mod p, else 0.
    Computed in O(p) by squaring each element of (Z/pZ)*.
    """
    table = np.zeros(p, dtype=np.int8)
    for x in range(1, p):
        table[(x * x) % p] = 1
    return table

@njit(fastmath=True, cache=True)
def fast_ap_engine(p: int, qr_table: np.ndarray) -> int:
    """
    Return Frobenius trace a_p for E: y^2 = x^3 - 4x over F_p.
    
    Key identity (Theorem 1.3(iii)):  a_p = -S_p  where
        S_p = sum_{t in F_p} chi(t^3 - 4t).
    
    CM property: for inert primes (p = 3 mod 4), a_p = 0.
    """
    s_t = 0
    for t in range(p):
        val = (t * t * t - 4 * t) % p
        if val == 0:
            continue          # chi(0) = 0: skip
        s_t += 1 if qr_table[val] == 1 else -1
    return -s_t
\end{lstlisting}

\subsection{Module 3: Parallel Processing}

The main computational loop processes all primes up to $2 \times 10^6$
in parallel using Python's \texttt{multiprocessing} module.

\begin{lstlisting}[caption={Parallel prime processing pipeline.},
                   label={lst:parallel}]
def process_prime(p: int) -> Dict:
    """
    Compute all arithmetic quantities for a single prime p.
    
    Returns dict with keys: p, type, pisano_period, a_p,
                             norm_trace, weil_ratio.
    'split': p = 1 (mod 4)   ->  a_p in [-2sqrt(p), 2sqrt(p)]
    'inert': p = 3 (mod 4)   ->  a_p = 0  (CM property)
    """
    qr_table   = build_qr_table(p)
    a_p        = fast_ap_engine(p, qr_table)
    sqrt_p     = np.sqrt(p)
    pisano_len = get_pisano_period(p)
    return {
        "p":             p,
        "type":          "split" if p % 4 == 1 else "inert",
        "pisano_period": pisano_len,
        "a_p":           a_p,
        "norm_trace":    a_p / sqrt_p,
        "weil_ratio":    abs(a_p) / (2 * sqrt_p),
    }

# Parallelise over all primes up to PRIME_LIMIT
with Pool(processes=cpu_count()) as pool:
    results = list(tqdm(
        pool.imap(process_prime, primes, chunksize=512),
        total=len(primes),
        desc="Computing"
    ))
\end{lstlisting}

% ============================================================================
\section{Additional Figures}
\label{sec:S4-figures}
% ============================================================================

\begin{remark}
Figures S1--S3 below are the full-resolution (600 dpi) versions of the
figures appearing in the main paper. They are reproduced here at screen
resolution for convenience.
\end{remark}

\subsection{Figure S1: Frobenius Trace Scatter}

\begin{figure}[H]
\centering
\includegraphics[width=\textwidth]{Fig1_Trace_Analysis.png}
\caption{\textbf{(Figure S1)} Normalised Frobenius traces $a_p/\sqrt{p}$
for the CM elliptic curve $E\colon y^2 = x^3-4x$ over $\mathbb{F}_p$,
for all $148{,}932$ primes $p \le 1{,}999{,}993$.
Split primes ($p\equiv 1\pmod{4}$, gray dots) are distributed across
$[-2, 2]$. Inert primes ($p\equiv 3\pmod{4}$, black markers) satisfy
$a_p = 0$ exactly, confirming the CM property.
The inset shows an enlarged view near $a_p/\sqrt{p} = 0$ for small
primes, highlighting the exact vanishing for all inert primes.}
\label{fig:S1}
\end{figure}

\subsection{Figure S2: CM Sato--Tate Distribution}

\begin{figure}[H]
\centering
\includegraphics[width=\textwidth]{Fig2_SatoTate.png}
\caption{\textbf{(Figure S2)} Empirical distribution of normalised
traces $a_p/\sqrt{p}$ for split primes, compared with the theoretical
CM Sato--Tate density $(\pi\sqrt{4-x^2})^{-1}$ (dashed curve).
The arcsine law arises from the CM structure of $E$ by $\mathbb{Z}[i]$:
split primes $p\equiv 1\pmod{4}$ factor as $p = \pi\bar\pi$ in
$\mathbb{Z}[i]$, and $a_p = \pi + \bar\pi$ for the canonical Hecke
Gr\"{o}ssencharacter.}
\label{fig:S2}
\end{figure}

\subsection{Figure S3: Chebotarev Density Convergence}

\begin{figure}[H]
\centering
\includegraphics[width=\textwidth]{Fig3_Convergence.png}
\caption{\textbf{(Figure S3)} Cumulative proportion of inert primes
($p\equiv 3\pmod{4}$) as a function of $p$, converging to the
Chebotarev density $1/2$ as $p\to\infty$.
The empirical ratio for $p \le 1{,}999{,}993$ is $0.500336$,
within $0.07\%$ of the theoretical prediction $0.500000$.}
\label{fig:S3}
\end{figure}

% ============================================================================
\section*{Acknowledgments}
% ============================================================================

The author thanks the anonymous referees for their careful reading
and valuable suggestions.

\end{document}
